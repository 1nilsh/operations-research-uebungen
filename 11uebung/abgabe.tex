\documentclass[a4paper,11pt]{article}

\usepackage[utf8]{inputenc} % Unicode support (Umlauts etc.)
\usepackage[ngerman]{babel} % Change hyphenation rules
\usepackage{ziffer} % , können in Zahlen verwendet werden ohne Formatierung kaputt zu machen
\usepackage[top=30mm,right=20mm,bottom=15mm,left=25mm,includefoot,headheight=32pt]{geometry} % Seitenränder

\usepackage{lmodern,textcomp,amsfonts} % The package supports the Text Companion fonts, which provide many text symbols
\usepackage[fleqn]{amsmath} % Formatierte Gleichungen
\usepackage{graphicx} % Grafiken
\usepackage{xcolor} % Farbe in Text
\usepackage{fancyhdr} % Seitenstil mit Kopfzeile etc.
\usepackage{cases} % Fallunterscheidungen mathematisch übereinander

\usepackage{tikz} % Graphen
\usetikzlibrary{positioning}

\pagestyle{fancy}
\fancyhf{}
\lhead{
    Lösung \\
    Übungsblatt 11
}
\rhead{Gruppe 3 \\Nils \textbf{Hodys}, Sascha \textbf{Majewsky}}
\rfoot{Seite \thepage}

\setlength{\parindent}{0cm} % Keine Einrückung der 1. Zeile eines Absatzes

\begin{document}

\raggedright % Alles Linksbündig
\setlength{\mathindent}{0cm} % align* nicht eingerückt

\section*{Aufgabe 1}

\textbf{Tour}: Die angefahrenen Kunden, ohne, dass deren Reihenfolge beachtet wird. \\
\textbf{Route}: Die Reihenfolge der eben erklärten Tour. \\
\textbf{Depot}: Die Quelle und Senke im System, an der Routen beginnen/enden. \\

\section*{Aufgabe 2}

\subsection*{a)}

\begin{align*}
    N &= \{A, B, C, D, E, F\} \\
    A &= \{ \\
      &~(B, A), (C, A), (D, A), (F, A), \\
      &~(A, B), (C, B), (D, B), (E, B), (F, B), \\
      &~(B, C), (A, C), (E, C), (F, C), \\
      &~(C, D), (A, D), (E, D), (F, D), \\
      &~(B, E), (C, E), (D, E), (A, E), (F, E), \\
      &~(B, F), (D, F), (E, F), (A, F) \\
        \} \\
    G &= (N, A): ~\text{Gerichteter Graph G, der das TSP darstellt} \\
    c_{ij} &: ~\text{Kosten der Kante ij} \\
    x_{ij} & \left\{\begin{array}{l}
        1: ~\text{Kante von i zu j gehört zur Tour} \\
        0: ~\text{Kante von i zu j gehört nicht zur Tour} \\
    \end{array}\right. \\
\end{align*}

\subsubsection*{LP: }

\begin{align*}
    \text{min. } z  =&~ 2x_{AB} 
                    + 2x_{AC} 
                    + 3x_{AD} 
                    + 4x_{AF} \\
                    &+ 2x_{BA} 
                    + 2x_{BC} 
                    + 3x_{BD} 
                    + 4x_{BE} 
                    + 6x_{BF} \\
                    &+ 2x_{CA} 
                    + 3x_{CB} 
                    + 3x_{CE} 
                    + 12x_{CF} \\
                    &+ 3x_{DA} 
                    + 4x_{DC} 
                    + 5x_{DE} 
                    + 3x_{DF} \\
                    &+ 4x_{EA} 
                    + 4x_{EB} 
                    + 3x_{EC} 
                    + 5x_{ED} 
                    + 2x_{EF} \\
                    &+ 4x_{FA} 
                    + 6x_{FB} 
                    + 3x_{FD} 
                    + 4x_{FE}
\end{align*}
\begin{align*}
    \text{s.t. } & x_{AB} + x_{AC} + x_{AD} + x_{AF} = 1 \\ % A ->
                 & x_{BA} + x_{CA} + x_{DA} + x_{EA} + x_{FA} = 1 \\ % A <-
                 & x_{BA} + x_{BC} + x_{BD} + x_{BE} + x_{BF} = 1 \\ % B ->
                 & x_{AB} + x_{CB} + x_{EB} + x_{FB} = 1 \\ % B <-
                 & x_{CA} + x_{CB} + x_{CE} + x_{CF} = 1 \\ % C ->
                 & x_{AC} + x_{BC} + x_{DC} + x_{EC} = 1 \\ % C <-
                 & x_{DA} + x_{DC} + x_{DE} + x_{DF} = 1 \\ % D ->
                 & x_{AD} + x_{BD} + x_{ED} + x_{FD} = 1 \\ % D <-
                 & x_{EA} + x_{EB} + x_{EC} + x_{ED} = 1 \\ % E ->
                 & x_{BE} + x_{CE} + x_{DE} + x_{FE} = 1 \\ % E <-
                 & x_{FA} + x_{FB} + x_{FD} + x_{FE} = 1 \\ % F ->
                 & x_{AF} + x_{BF} + x_{CF} + x_{DF} + x_{EF} = 1 \\ % F <-
                 & x_{ij} \in \{1, 0\}
\end{align*}

\subsection*{b)}

Sollen hier Beispiele aus der echten Welt hin?

\subsection*{c)}

Bei symmetrischem Graph: \\
\begin{align*}
    |s| &= \frac{n!}{2n} \\
        &= \frac{6!}{2*6} \\
        &= 60 \\
\end{align*}

Bei asymmetrischem Graph: \\
\begin{align*}
    |s| &= n! \\
        &= 720 \\
\end{align*}

\section*{Aufgabe 3}


\end{document}