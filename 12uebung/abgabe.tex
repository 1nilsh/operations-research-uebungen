\documentclass[a4paper,11pt]{article}

\usepackage[utf8]{inputenc} % Unicode support (Umlauts etc.)
\usepackage[ngerman]{babel} % Change hyphenation rules
\usepackage{ziffer} % , können in Zahlen verwendet werden ohne Formatierung kaputt zu machen
\usepackage[top=30mm,right=20mm,bottom=15mm,left=25mm,includefoot,headheight=32pt]{geometry} % Seitenränder

\usepackage{lmodern,textcomp,amsfonts} % The package supports the Text Companion fonts, which provide many text symbols
\usepackage[fleqn]{amsmath} % Formatierte Gleichungen
\usepackage{graphicx} % Grafiken
\usepackage{xcolor} % Farbe in Text
\usepackage{fancyhdr} % Seitenstil mit Kopfzeile etc.
\usepackage{cases} % Fallunterscheidungen mathematisch übereinander

\usepackage{fontawesome}

\usepackage{tikz} % Graphen
\usetikzlibrary{positioning}

\usepackage{tcolorbox}

\newtcbox{\wbox}{left=0pt, right=0pt, top=0pt, bottom=0pt, arc=0pt, boxrule=0pt, colback=white, colframe=white}

\pagestyle{fancy}
\fancyhf{}
\lhead{
    Lösung \\
    Übungsblatt 12
}
\rhead{Gruppe 3 \\Nils \textbf{Hodys}, Sascha \textbf{Majewsky}}
\rfoot{Seite \thepage}

\setlength{\parindent}{0cm} % Keine Einrückung der 1. Zeile eines Absatzes

\begin{document}

\raggedright % Alles Linksbündig
\setlength{\mathindent}{0cm} % align* nicht eingerückt

\section*{Aufgabe 1}

\subsection*{Schritt 1: Knoten sortieren}

\begin{tikzpicture}[inner sep=3mm, node distance=5cm and 2cm, thick, main/.style = {draw, circle}]
    \node[main] (D) [] { D };
    \node[main] (1) [above right of=D] { 1 };
    \node[main] (2) [above left of=D] { 2 };
    \node[main] (3) [left of=D] { 3 };
    \node[main] (4) [below of=D] { 4 };
    \node[main] (5) [right of=D] { 5 };
  
    \draw[-] (D) -- node[midway] {\wbox{4}} (1);
    \draw[-] (D) -- node[midway] {\wbox{3}} (2); % siehe mystudy
    \draw[-] (D) -- node[midway] {\wbox{3}} (3);
    \draw[-] (D) -- node[midway] {\wbox{2}} (4);
    \draw[-] (D) -- node[midway] {\wbox{2}} (5);

    \draw[-] (1) -- node[midway] {\wbox{1}} (2);
    \draw[-] (1) -- node[midway] {\wbox{5}} (3);
    \draw[-] (1) -- node[midway] {\wbox{6}} (4);
    \draw[-] (1) -- node[midway] {\wbox{4}} (5);

    \draw[-] (2) -- node[midway] {\wbox{2}} (3);
    \draw[-] (2) -- node[midway] {\wbox{6}} (4);
    \draw[-] (2) -- node[midway] {\wbox{7}} (5);

    \draw[-] (3) -- node[midway] {\wbox{3}} (4);

    \draw[-] (4) -- node[midway] {\wbox{2}} (5);
\end{tikzpicture} \\

\subsection*{Schritt 2: Tourenplan bilden}

Tourenplan 1: \\
$[1,2][3,4][5]$ \\
Gesamtlänge: 20 \\

Tourenplan 2: \\
$[2,3][4,5][1]$ \\
Gesamtlänge: 22 \\

Tourenplan 3: \\
$[3,4][5,1][2]$ \\
Gesamtlänge: 24 \\

Tourenplan 4: \\
$[4,5][1,2][3]$ \\
Gesamtlänge: 19 \\

Tourenplan 5: \\
$[5,1][2,3][4]$ \\
Gesamtlänge: 22 \newline

Der Kürzeste Tourenplan ist Tourenplan 4. \\
$D \to 4 \to 5 \to D \to 1 \to 2 \to D \to 3 \to D$

\section*{Aufgabe 2}

\section*{Aufgabe 3}

\end{document}