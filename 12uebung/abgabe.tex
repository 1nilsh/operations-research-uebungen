\documentclass[a4paper,11pt]{article}

\usepackage[utf8]{inputenc} % Unicode support (Umlauts etc.)
\usepackage[ngerman]{babel} % Change hyphenation rules
\usepackage{ziffer} % , können in Zahlen verwendet werden ohne Formatierung kaputt zu machen
\usepackage[top=30mm,right=20mm,bottom=15mm,left=25mm,includefoot,headheight=32pt]{geometry} % Seitenränder

\usepackage{lmodern,textcomp,amsfonts} % The package supports the Text Companion fonts, which provide many text symbols
\usepackage[fleqn]{amsmath} % Formatierte Gleichungen
\usepackage{graphicx} % Grafiken
\usepackage{xcolor} % Farbe in Text
\usepackage{fancyhdr} % Seitenstil mit Kopfzeile etc.
\usepackage{cases} % Fallunterscheidungen mathematisch übereinander

\usepackage{fontawesome}

\usepackage{tikz} % Graphen
\usetikzlibrary{positioning}

\usepackage{tcolorbox}

\newtcbox{\wbox}{left=0pt, right=0pt, top=0pt, bottom=0pt, arc=0pt, boxrule=0pt, colback=white, colframe=white}

\pagestyle{fancy}
\fancyhf{}
\lhead{
    Lösung \\
    Übungsblatt 12
}
\rhead{Gruppe 3 \\Nils \textbf{Hodys}, Sascha \textbf{Majewsky}}
\rfoot{Seite \thepage}

\setlength{\parindent}{0cm} % Keine Einrückung der 1. Zeile eines Absatzes

\begin{document}

\raggedright % Alles Linksbündig
\setlength{\mathindent}{0cm} % align* nicht eingerückt

\section*{Aufgabe 1}

\subsection*{Schritt 1: Knoten sortieren}

\begin{tikzpicture}[inner sep=3mm, node distance=5cm and 2cm, thick, main/.style = {draw, circle}]
    \node[main] (D) [] { D };
    \node[main] (1) [above right of=D] { 1 };
    \node[main] (2) [above left of=D] { 2 };
    \node[main] (3) [left of=D] { 3 };
    \node[main] (4) [below of=D] { 4 };
    \node[main] (5) [right of=D] { 5 };
  
    \draw[-] (D) -- node[midway] {\wbox{4}} (1);
    \draw[-] (D) -- node[midway] {\wbox{3}} (2); % siehe mystudy
    \draw[-] (D) -- node[midway] {\wbox{3}} (3);
    \draw[-] (D) -- node[midway] {\wbox{2}} (4);
    \draw[-] (D) -- node[midway] {\wbox{2}} (5);

    \draw[-] (1) -- node[midway] {\wbox{1}} (2);
    \draw[-] (1) -- node[midway] {\wbox{5}} (3);
    \draw[-] (1) -- node[midway] {\wbox{6}} (4);
    \draw[-] (1) -- node[midway] {\wbox{4}} (5);

    \draw[-] (2) -- node[midway] {\wbox{2}} (3);
    \draw[-] (2) -- node[midway] {\wbox{6}} (4);
    \draw[-] (2) -- node[midway] {\wbox{7}} (5);

    \draw[-] (3) -- node[midway] {\wbox{3}} (4);

    \draw[-] (4) -- node[midway] {\wbox{2}} (5);
\end{tikzpicture} \\

\subsection*{Schritt 2: Tourenplan bilden}

Tourenplan 1: \\
$[1,2][3,4][5]$ \\
Gesamtlänge: 20 \newline

Tourenplan 2: \\
$[2,3][4,5][1]$ \\
Gesamtlänge: 22 \newline

Tourenplan 3: \\
$[3,4][5,1][2]$ \\
Gesamtlänge: 24 \newline

Tourenplan 4: \\
$[4,5][1,2][3]$ \\
Gesamtlänge: 20 \newline

Tourenplan 5: \\
$[5,1][2,3][4]$ \\
Gesamtlänge: 22 \newline

Die kürzesten Tourenpläne sind Tourenplan 1 und 4, die beide eine Gesamtlänge von 20 haben. \\
Tourenplan 1: $D \to 1 \to 2 \to D \to 3 \to 4 \to D \to 5 \to D$ \\
Tourenplan 4: $D \to 4 \to 5 \to D \to 1 \to 2 \to D \to 3 \to D$ \\

\section*{Aufgabe 2}

\subsection*{Teil 1}

$D \leftrightarrow K_1: 2 * 20\text{km}$ \\
$D \leftrightarrow K_2: 2 * 30\text{km}$ \\
$D \leftrightarrow K_3: 2 * 40\text{km}$ \\
$D \leftrightarrow K_4: 2 * 30\text{km}$ \\
$D \leftrightarrow K_5: 2 * 40\text{km}$ \\
Tourenlänge insgesamt: 320km, entspricht 4800€

\subsection*{Teil 2}

Savings-Matrix: \newline

\begin{tabular}{c|c|c|c|c|c}
   &  K1  &  K2  &  K3  &  K4  &  K5  \\ \hline
K1 &  -   &  40  &  10  & -10  &  10  \\ \hline
K2 &      &  -   &  40  &  10  &  10  \\ \hline
K3 &      &      &  -   &  50  &  30  \\ \hline
K4 &      &      &      &  -   &  30  \\ \hline
K5 &      &      &      &      &  - 
\end{tabular}

\subsection*{Teil 3}

\textbf{Schritt 1:} \\
K3-K4 Zusammenlegen. Neue Tourenlänge: 270km \newline

\textbf{Schritt 2:} \\
K1-K2 Zusammenlegen. Neue Tourenlänge: 230km \newline

\textbf{Schritt 3:} \\
K2-K3 Zusammenlegen nicht möglich aufgrund von Gewichtsbegrenzung. Tourenlänge weiterhin: 230km \newline

\textbf{Schritt 4:} \\
K5-K3 Zusammenlegen. Neue Tourenlänge: 200km \newline

\textbf{Schritt 5:} \\
K4-K5 Zusammenlegen nicht möglich, da diese schon Teil einer gemeinsamen Tour sind. Tourenlänge weiterhin: 200km \newline

\textbf{Schritt 6:} \\
K1-K3 Zusammenlegen nicht möglich aufgrund von Gewichtsbegrenzung. Tourenlänge weiterhin: 200km \newline

\textbf{Schritt 7:} \\
K1-K4 Zusammenlegen nicht sinnvoll aufgrund von negativem Saving. Tourenlänge weiterhin: 200km \newline

\textbf{Schritt 8:} \\
K1-K5 Zusammenlegen nicht möglich aufgrund von Gewichtsbegrenzung. Tourenlänge weiterhin: 200km \newline

\textbf{Schritt 9:} \\
K2-K4 Zusammenlegen nicht möglich aufgrund von Gewichtsbegrenzung. Tourenlänge weiterhin: 200km \newline

\textbf{Schritt 10:} \\
K2-K5 Zusammenlegen nicht möglich aufgrund von Gewichtsbegrenzung. Tourenlänge weiterhin: 200km \newline

\subsection*{Finale Touren}

\begin{tikzpicture}[inner sep=3mm, node distance=5cm and 2cm, thick, main/.style = {draw, circle}]
    \node[main] (D) [] { \textbf{D} };
    \node[main] (1) [right of=D] { 1 };
    \node[main] (2) [above of=D] { 2 };
    \node[main] (3) [left of=D] { 3 };
    \node[main] (4) [below left of=D] { 4 };
    \node[main] (5) [below right of=D] { 5 };
  
    \draw[->] (D) -- node[midway] {\wbox{20}} (1);
    \draw[->] (1) -- node[midway] {\wbox{10}} (2);
    \draw[->] (2) -- node[midway] {\wbox{30}} (D);

    \draw[->] (D) -- node[midway] {\wbox{40}} (5);
    \draw[->] (5) -- node[midway] {\wbox{50}} (3);
    \draw[->] (3) -- node[midway] {\wbox{20}} (4);
    \draw[->] (4) -- node[midway] {\wbox{30}} (D);
\end{tikzpicture} \newline

Länge der Touren: 200km = 3000€ \\

\section*{Aufgabe 3}

\subsection*{Anfangslösung}
\begin{tikzpicture}[inner sep=3mm, node distance=5cm and 2cm, thick, main/.style = {draw, circle}]
    \node[main] (D) [] { \textbf{D} };
    \node[main] (1) [above of=D] { 1 };
    \node[main] (2) [left of=D] { 2 };
    \node[main] (3) [below left of=D] { 3 };
    \node[main] (4) [below of=D] { 4 };
    \node[main] (5) [right of=D] { 5 };
    \node[main] (6) [above right of=D] { 6 };
  
    \draw[->] (D) -- node[midway] {\wbox{35}} (3);
    \draw[->] (3) -- node[midway] {\wbox{15}} (2);
    \draw[->] (2) -- node[midway] {\wbox{30}} (1);
    \draw[->] (1) -- node[midway] {\wbox{20}} (6);
    \draw[->] (6) -- node[midway] {\wbox{15}} (5);
    \draw[->] (5) -- node[midway] {\wbox{30}} (4);
    \draw[->] (4) -- node[midway] {\wbox{15}} (D);
\end{tikzpicture} \\

\textbf{Knoten durchnummerieren:} \newline

\begin{tabular}{c|c|c|c|c|c|c|c}
    D & 3 & 2 & 1 & 6 & 5 & 4 & D  \\ \hline
    1 & 2 & 3 & 4 & 5 & 6 & 7 & - 
\end{tabular} \newline

$i=1; j=3$ \\
Prüfung $ C_{D,3} + C_{2,1} > C_{D,2} + C_{3,1} $ \\
$\Leftrightarrow 35 + 30 > 20 + 15$ Wahr! \newline

Daher Tausch D,3 und 2,1 gegen D,2 und 3,1

\subsection*{Nächster Schritt}
\begin{tikzpicture}[inner sep=3mm, node distance=5cm and 2cm, thick, main/.style = {draw, circle}]
    \node[main] (D) [] { \textbf{D} };
    \node[main] (1) [above of=D] { 1 };
    \node[main] (2) [left of=D] { 2 };
    \node[main] (3) [below left of=D] { 3 };
    \node[main] (4) [below of=D] { 4 };
    \node[main] (5) [right of=D] { 5 };
    \node[main] (6) [above right of=D] { 6 };
  
    \draw[->] (D) -- node[midway] {\wbox{20}} (2);
    \draw[->] (2) -- node[midway] {\wbox{15}} (3);
    \draw[->] (3) -- node[midway] {\wbox{15}} (1);
    \draw[->] (1) -- node[midway] {\wbox{20}} (6);
    \draw[->] (6) -- node[midway] {\wbox{15}} (5);
    \draw[->] (5) -- node[midway] {\wbox{30}} (4);
    \draw[->] (4) -- node[midway] {\wbox{15}} (D);
\end{tikzpicture} \\

\textbf{Knoten durchnummerieren:} \newline

\begin{tabular}{c|c|c|c|c|c|c|c}
    D & 2 & 3 & 1 & 6 & 5 & 4 & D  \\ \hline
    1 & 2 & 3 & 4 & 5 & 6 & 7 & - 
\end{tabular} \newline

$i=1; j=3$ \\
Prüfung $ C_{D,2} + C_{3,1} > C_{D,3} + C_{2,1} $ \\
$\Leftrightarrow 20 + 15 > 35 + 30$ Falsch! \newline

$i=1; j=4$ \\
Prüfung $ C_{D,2} + C_{1,6} > C_{D,1} + C_{2,6} $ \\
$\Leftrightarrow 20 + 20 > 15 + \infty $ Falsch! \newline

$i=1; j=5$ \\
Prüfung $ C_{D,2} + C_{6,5} > C_{D,6} + C_{2,5} $ \\
$\Leftrightarrow 20 + 15 > 35 + 40 $ Falsch! \newline

$i=1; j=6$ \\
Prüfung $ C_{D,2} + C_{5,4} > C_{D,5} + C_{2,4} $ \\
$\Leftrightarrow 20 + 30 > 20 + 30 $ Falsch! \newline

$i=2; j=4$ \\
Prüfung $ C_{2,3} + C_{1,6} > C_{2,1} + C_{3,6} $ \\
$\Leftrightarrow 15 + 20 > 30 + 2 $ Wahr! \newline

Daher Tausch 2,3 und 1,6 gegen 2,1 und 3,6

\subsection*{Nächster Schritt}
\begin{tikzpicture}[inner sep=3mm, node distance=5cm and 2cm, thick, main/.style = {draw, circle}]
    \node[main] (D) [] { \textbf{D} };
    \node[main] (1) [above of=D] { 1 };
    \node[main] (2) [left of=D] { 2 };
    \node[main] (3) [below left of=D] { 3 };
    \node[main] (4) [below of=D] { 4 };
    \node[main] (5) [right of=D] { 5 };
    \node[main] (6) [above right of=D] { 6 };
  
    \draw[->] (D) -- node[midway] {\wbox{20}} (2);
    \draw[->] (2) -- node[midway] {\wbox{30}} (1);
    \draw[->] (1) -- node[midway] {\wbox{15}} (3);
    \draw[->] (3) to [looseness=.7] node[midway] {\wbox{2}} (6);
    \draw[->] (6) -- node[midway] {\wbox{15}} (5);
    \draw[->] (5) -- node[midway] {\wbox{30}} (4);
    \draw[->] (4) -- node[midway] {\wbox{15}} (D);
\end{tikzpicture} \newline

\textit{Abbruch nach 2 mal tauschen.} \newline

Endlösung: $D \to 2 \to 1 \to 3 \to 6 \to 5 \to 4 \to D$ \\

Länge der Lösung: 127 \\

\end{document}