\documentclass[a4paper,11pt]{article}

\usepackage[utf8]{inputenc} % Unicode support (Umlauts etc.)
\usepackage[ngerman]{babel} % Change hyphenation rules
\usepackage{ziffer} % , können in Zahlen verwendet werden ohne Formatierung kaputt zu machen
\usepackage[top=30mm,right=20mm,bottom=15mm,left=25mm,includefoot,headheight=32pt]{geometry} % Seitenränder

\usepackage{lmodern,textcomp} % The package supports the Text Companion fonts, which provide many text symbols (benötigt für €)
\usepackage[fleqn]{amsmath} % Formatierte Gleichungen
\usepackage{graphicx} % Grafiken
\usepackage{xcolor} % Farbe in Text
\usepackage{fancyhdr} % Seitenstil mit Kopfzeile etc.
\usepackage{cases} % Fallunterscheidungen mathematisch uebereinander

%%%%%%%%%%%%%%%%%%%%%%%%%%
% Sourcecode Darstellung %
%%%%%%%%%%%%%%%%%%%%%%%%%%

\usepackage{listings}
\usepackage{xcolor}

\renewcommand\lstlistingname{Quellcode} % Bezeichnung unter einzelnem Listing
\renewcommand\lstlistlistingname{Quellcodeverzeichnis} % Überschrift des Verzeichnisses

\definecolor{codegreen}{rgb}{0,0.6,0}
\definecolor{codegray}{rgb}{0.5,0.5,0.5}
\definecolor{codepurple}{rgb}{0.58,0,0.82}
\definecolor{backcolour}{rgb}{0.99,0.99,0.99}

\lstdefinestyle{mystyle}{
    backgroundcolor=\color{backcolour},   
    commentstyle=\color{codegreen},
    keywordstyle=\color{magenta},
    numberstyle=\tiny\color{codegray},
    stringstyle=\color{codepurple},
    basicstyle=\ttfamily\footnotesize,
    breakatwhitespace=false,         
    breaklines=true,                 
    captionpos=b,                    
    keepspaces=true,                 
    numbers=left,                    
    numbersep=5pt,                  
    showspaces=false,                
    showstringspaces=false,
    showtabs=false,                  
    tabsize=2
}

\lstset{style=mystyle} % Oben definierten Style als Standard



\pagestyle{fancy}
\fancyhf{}
\lhead{
    Lösung \\
    Übungsblatt 8
}
\rhead{Gruppe 3 \\Nils \textbf{Hodys}, Sascha \textbf{Majewsky}}
\rfoot{Seite \thepage}

\setlength{\parindent}{0cm} % Keine Einrückung der 1. Zeile eines Absatzes

\begin{document}

\raggedright % Alles Linksbündig

\section*{Aufgabe 1}

\subsection*{Indexmenge}

$P:$ Menge der Produkte \\

\subsection*{Parameter}

\begin{align*}
  d_p :&~ \text{Umsatz mit Produkt $p \in P$} \\
  h_p :&~ \text{Benötigtes Holz für Produkt $p \in P$} \\
  s_p :&~ \text{Schreinerkosten für Produkt $p \in P$} \\
  l_p :&~ \text{Lackiererkosten für Produkt $p \in P$} \\
  h_{max} :&~ \text{Höchstmenge Holz/Woche (kg)}
\end{align*}

\subsection*{Entscheidungsvariablen}

\begin{align*}
  x_p :&~ \text{Produktion pro Woche von Produkt $p \in P$}
\end{align*}

\subsection*{Zielfunktion}

\begin{align*}
  \text{max.} ~ \sum x_p * (d_p - s_p - l_p)
\end{align*}

\subsection*{Restriktionen}

\begin{align*}
  \forall p \in P : \sum x_p * h_p &\le h_{max} &&~\big|~ \text{Höchstmenge Holz/Woche} \\
  \forall p \in P : x_p &\ge 0 &&~\big|~ \text{NNB} \\
\end{align*}

\subsection*{AMPL}

\subsubsection*{.mod Datei}

\begin{lstlisting}[]
  set P;
  param d{P};
  param h{P};
  param s{P};
  param l{P};
  param hmax;
  var y{P};

  maximize profit: sum{p in P} (x[p] * (d[p] - s[p] - l[p]));
  subject to maxHolz: sum{p in P} x[p] * h[p] <= hmax;
\end{lstlisting}

\vspace{10mm}

\subsubsection*{.dat Datei}

\begin{lstlisting}[]

\end{lstlisting}

\end{document}