\documentclass[a4paper,11pt]{article}

\usepackage[utf8]{inputenc} % Unicode support (Umlauts etc.)
\usepackage[ngerman]{babel} % Change hyphenation rules
\usepackage{ziffer} % , können in Zahlen verwendet werden ohne Formatierung kaputt zu machen
\usepackage[top=30mm,right=20mm,bottom=15mm,left=25mm,includefoot,headheight=32pt]{geometry} % Seitenränder

\usepackage{lmodern,textcomp,amsfonts} % The package supports the Text Companion fonts, which provide many text symbols
\usepackage[fleqn]{amsmath} % Formatierte Gleichungen
\usepackage{graphicx} % Grafiken
\usepackage{xcolor} % Farbe in Text
\usepackage{fancyhdr} % Seitenstil mit Kopfzeile etc.
\usepackage{cases} % Fallunterscheidungen mathematisch übereinander

\usepackage{multicol} % Zweispaltiges Layout

\pagestyle{fancy}
\fancyhf{}
\lhead{
    Lösung \\
    Übungsblatt 9
}
\rhead{Gruppe 3 \\Nils \textbf{Hodys}, Sascha \textbf{Majewsky}}
\rfoot{Seite \thepage}

\setlength{\parindent}{0cm} % Keine Einrückung der 1. Zeile eines Absatzes

\begin{document}

\raggedright % Alles Linksbündig
\setlength{\mathindent}{0cm} % align* nicht eingerückt

\section*{Aufgabe 1}

{\setlength{\columnseprule}{.1pt}
\setlength{\columnsep}{3cm}
\begin{multicols}{2}

\textbf{Step 0} \\
\begin{tabular}{ |c|c| } 
  \hline
  B & \{ A \} \\
  \hline
  S & \{  \} \\ 
  \hline
 \end{tabular} \\
\begin{tabular}{ |c|c|c|c|c|c|c|c| } 
  \hline
       & A & B & C & D & E & F & G \\
  \hline
  d(j) & 0 & $\infty$ & $\infty$ & $\infty$ & $\infty$ & $\infty$ & $\infty$ \\
  \hline
  v(j) & - &  &  &  &  &  &  \\
  \hline
\end{tabular}
\vspace{4mm}

\textbf{Step 1} \\
\begin{tabular}{ |c|c| } 
  \hline
  B & \{ B, C, D \} \\
  \hline
  S & \{ A \} \\ 
  \hline
 \end{tabular} \\
\begin{tabular}{ |c|c|c|c|c|c|c|c| } 
  \hline
       & A & B & C & D & E & F & G \\
  \hline
  d(j) & 0 & 5 & 3 & 2 & $\infty$ & $\infty$ & $\infty$ \\
  \hline
  v(j) & - & A & A & A &  &  &  \\
  \hline
\end{tabular}
\vspace{4mm}

Günstigstes d(j): D \\

\textbf{Step 2} \\
\begin{tabular}{ |c|c| } 
  \hline
  B & \{ B, C \} \\
  \hline
  S & \{ A, D \} \\ 
  \hline
 \end{tabular} \\
\begin{tabular}{ |c|c|c|c|c|c|c|c| } 
  \hline
       & A & B & C & D & E & F & G \\
  \hline
  d(j) & 0 & 5 & 3 & 2 & $\infty$ & $\infty$ & $\infty$ \\
  \hline
  v(j) & - & A & A & A &  &  &  \\
  \hline
\end{tabular}
\vspace{4mm}

Günstigstes d(j): C \\

\textbf{Step 3} \\
\begin{tabular}{ |c|c| } 
  \hline
  B & \{ B, E \} \\
  \hline
  S & \{ A, D, C \} \\ 
  \hline
 \end{tabular} \\
\begin{tabular}{ |c|c|c|c|c|c|c|c| } 
  \hline
       & A & B & C & D & E & F & G \\
  \hline
  d(j) & 0 & 4 & 3 & 2 & 5 & $\infty$ & $\infty$ \\
  \hline
  v(j) & - & C & A & A & C &  &  \\
  \hline
\end{tabular}
\vspace{4mm}

Günstigstes d(j): B \\

\columnbreak

\textbf{Step 4} \\
\begin{tabular}{ |c|c| } 
  \hline
  B & \{ E, F, G \} \\
  \hline
  S & \{ A, D, C, B \} \\ 
  \hline
 \end{tabular} \\
\begin{tabular}{ |c|c|c|c|c|c|c|c| } 
  \hline
       & A & B & C & D & E & F & G \\
  \hline
  d(j) & 0 & 4 & 3 & 2 & 5 & 7 & 5 \\
  \hline
  v(j) & - & C & A & A & C & B & B \\
  \hline
\end{tabular}
\vspace{4mm}

Günstigstes d(j): E \\

\textbf{Step 5} \\
\begin{tabular}{ |c|c| } 
  \hline
  B & \{ F, G \} \\
  \hline
  S & \{ A, D, C, B, E \} \\ 
  \hline
 \end{tabular} \\
\begin{tabular}{ |c|c|c|c|c|c|c|c| } 
  \hline
       & A & B & C & D & E & F & G \\
  \hline
  d(j) & 0 & 4 & 3 & 2 & 5 & 6 & 5 \\
  \hline
  v(j) & - & C & A & A & C & E & B \\
  \hline
\end{tabular}
\vspace{4mm}

Günstigstes d(j): G \\

\textbf{Step 6} \\
\begin{tabular}{ |c|c| } 
  \hline
  B & \{ F \} \\
  \hline
  S & \{ A, D, C, B, E, G \} \\ 
  \hline
 \end{tabular} \\
\begin{tabular}{ |c|c|c|c|c|c|c|c| } 
  \hline
       & A & B & C & D & E & F & G \\
  \hline
  d(j) & 0 & 4 & 3 & 2 & 5 & 6 & 5 \\
  \hline
  v(j) & - & C & A & A & C & E & B \\
  \hline
\end{tabular}
\vspace{4mm}

Günstigstes d(j): F \\

\textbf{Step 7} \\
\begin{tabular}{ |c|c| } 
  \hline
  B & \{ \} \\
  \hline
  S & \{ A, D, C, B, E, G, F \} \\ 
  \hline
 \end{tabular} \\
\begin{tabular}{ |c|c|c|c|c|c|c|c| } 
  \hline
       & A & B & C & D & E & F & G \\
  \hline
  d(j) & 0 & 4 & 3 & 2 & 5 & 6 & 5 \\
  \hline
  v(j) & - & C & A & A & C & E & B \\
  \hline
\end{tabular}
\vspace{4mm}

Die Menge B ist nun Leer; \\
$\to$ Abbruchbedingung.

\end{multicols}}
\vspace{10mm}


\section*{Aufgabe 2}

\subsection*{Entscheidungsvariablen}
$x_{i j}$ : Menge der über die Kante $(i,j) \in A$ zu transportierenden Einheiten (in Stk.), wobei A die Menge aller Kanten
\begin{multline*}
A = \{(v_1, v_2),(v_1, u_1),(v_1, k_1),(v_1, u_2),(v_2, u_1),\\
(v_2, k_3),(v_2, u_3),(u_1, k_1),(u_2, k_1),(u_2, k_2),(u_3, k_1),(u_3, k_2),(u_3, k_3),(k_1, k_2),(k_2, k_3)\}
\end{multline*}
ist. \\
\subsection*{Zielfunktion}
\begin{multline*}
  \text{min. } z =~ 
      1x_{v_1 v_2}
    + 2x_{v_1 u_1}
    + 3x_{v_1 k_1}
    + 1x_{v_1 u_2}
    + 2x_{v_2 u_1}
    + 4x_{v_2 k_3} \\
    + 2x_{v_2 u_3}
    + 2x_{u_1 k_1}
    + 1x_{u_2 k_1}
    + 3x_{u_2 k_2}
    + 5x_{u_3 k_1}
    + 2x_{u_3 k_2}
    + 1x_{u_3 k_3}
    + 1x_{k_1 k_2}
    + 2x_{k_2 k_3}
\end{multline*}

\subsection*{Restriktionen}
\begin{align*}
\text{s.t. }
& x_{v_1 v_2} + x_{v_1 u_1} + x_{v_1 k_1} + x_{v_1 u_2} = 20 \\ % Abflüsse v1
& x_{v_2 u_1} + x_{v_2 k_3} + x_{v_2 u_3} - x_{v_1 v_2} = 10 \\ % Abflüsse - Zufluss v2
& x_{v_1 u_1} + x_{v_2 u_1} - x_{u_1 k_1} = 0 \\ % Zuflüsse - Abflüsse u1
& x_{v_1 u_2} - x_{u_2 k_1} - x_{u_2 k_2} = 0 \\ % Zuflüsse - Abflüsse u2
& x_{v_2 u_3} - x_{u_3 k_1} - x_{u_3 k_2} - x_{u_3 k_3} = 0 \\ % Zuflüsse - Abflüsse u3
& x_{u_1 k_1} + x_{v_1 k_1} + x_{u_2 k_1} + x_{u_3 k_1} - x_{k_1 k_2} = 5 \\ % Zuflüsse - Abflüsse k1
& x_{u_2 k_2} + x_{u_3 k_2} + x_{k_1 k_2} - x_{k_2 k_3} = 15 \\ % Zuflüsse - Abflüsse k2
& x_{v_2 k_3} + x_{u_3 k_3} + x_{k_2 k_3} = 10 \\ % Zuflüsse k3
& x_{v_1 v_2} \le 8 \\ % Durchflüsse
& x_{v_1 u_1} \le 9 \\ % Durchflüsse
& x_{v_1 k_1} \le 6 \\ % Durchflüsse
& x_{v_1 u_2} \le 5 \\ % Durchflüsse
& x_{v_2 u_1} \le 7 \\ % Durchflüsse
& x_{v_2 k_3} \le 4 \\ % Durchflüsse
& x_{v_2 u_3} \le 5 \\ % Durchflüsse
& x_{u_1 k_1} \le 9 \\ % Durchflüsse
& x_{u_2 k_1} \le 6 \\ % Durchflüsse
& x_{u_2 k_2} \le 8 \\ % Durchflüsse
& x_{u_3 k_1} \le 4 \\ % Durchflüsse
& x_{u_3 k_2} \le 6 \\ % Durchflüsse
& x_{u_3 k_3} \le 4 \\ % Durchflüsse
& x_{k_1 k_2} \le 7 \\ % Durchflüsse
& x_{k_2 k_3} \le 3 \\ % Durchflüsse
& x_{v_1 v_2},
  x_{v_1 u_1},
  x_{v_1 k_1},
  x_{v_1 u_2},
  x_{v_2 u_1},
  x_{v_2 k_3},
  x_{v_2 u_3},
  x_{u_1 k_1},
  x_{u_2 k_1},
  x_{u_2 k_2},
  x_{u_3 k_1},
  x_{u_3 k_2},
  x_{u_3 k_3},
  x_{k_1 k_2},
  x_{k_2 k_3} \ge 0 \\ % NNB
& x_{v_1 v_2},
  x_{v_1 u_1},
  x_{v_1 k_1},
  x_{v_1 u_2},
  x_{v_2 u_1},
  x_{v_2 k_3},
  x_{v_2 u_3},
  x_{u_1 k_1},
  x_{u_2 k_1},
  x_{u_2 k_2},
  x_{u_3 k_1},
  x_{u_3 k_2},
  x_{u_3 k_3},
  x_{k_1 k_2},
  x_{k_2 k_3} \in \mathbb{Z} % integer
\end{align*}

\vspace{4mm}

Das Problem geht nicht auf, da der Bedarf von $k_2$ und $k_3$ bei den gegebenen Restriktionen nicht gleichzeitig gedeckt werden kann.

\end{document}