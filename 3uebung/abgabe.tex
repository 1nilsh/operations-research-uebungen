\documentclass[a4paper,11pt]{article}

\usepackage[utf8]{inputenc} % Unicode support (Umlauts etc.)
\usepackage[ngerman]{babel} % Change hyphenation rules
\usepackage{ziffer} % , können in Zahlen verwendet werden ohne Formatierung kaputt zu machen
\usepackage[top=25mm,right=25mm,bottom=20mm,left=25mm,includefoot,headheight=28pt]{geometry} % Seitenränder

\usepackage{lmodern,textcomp} % The package supports the Text Companion fonts, which provide many text symbols (benötigt für €)
\usepackage[fleqn]{amsmath} % Formatierte Gleichungen
\usepackage{graphicx} % Grafiken
\usepackage{xcolor} % Farbe in Text
\usepackage{fancyhdr} % Seitenstil mit Kopfzeile etc.

\pagestyle{fancy}
\fancyhf{}
\lhead{Lösung \\ Übungsblatt 3}
\rhead{Gruppe 8 \\ Michel \textbf{Hansen}, Pascal \textbf{Heinrich}, Nils \textbf{Hodys}, Sascha \textbf{Majewsky}}
\rfoot{Seite \thepage}

\setlength{\parindent}{0cm} % Keine Einrückung der 1. Zeile eines Absatzes

\begin{document}

\raggedright % Alles Linksbündig


\section*{Aufgabe 1}
\subsection*{Ausgangsmodell}
\begin{align*}
\text{max. } & z = 4x_1 - 2x_2 \\
\text{s.t. } & x_3 = 30 - x_1 - 2x_2 \\
& x_4 = {\color{red} -20} - 2x_1 + 2x_2  & \color{red} \to \text{Phase I notwendig} \\
& x_5 = 25 -4x_1 - x_2 \\
& x_1, x_2, x_3, x_4, x_5 \ge 0 \\
\end{align*}

\subsection*{Neue Zielfunktion \emph{s}}
\begin{align*}
\text{max. } s &= x_4 \\
&= -20 - 2x1 + {\color{red} 2x_2}
\end{align*}

\begin{align*}
& z = 0 + 4x_1 - 2x_2 \\
& x_3 = 30 - x_1 - 2x_2 &&\big| x_2 \uparrow 15 \\
& x_4 = -20 - 2x_1 + {\color{red} 2x_2} &&\big| \color{red} x_2 \uparrow 10 \\
& x_5 = 25 -4x_1 - x_2 &&\big| x_2 \uparrow 12,5 \\
& x_1, x_2, x_3, x_4, x_5 \ge 0 \\
\end{align*}

\subsection*{Nach Umformen und Einsetzen}
\begin{align*}
\text{max. } & s = x_4 \\
\text{s.t. } & z = -20 + 2x_1 - x_4 \\
& x_2 = 10 + x_1 + \frac{1}{2}x_4 \\
& x_3 = 10 - 3x_1 - x_4 \\
& x_5 = 15 - 5x_1 - \frac{1}{2}x_4 \\
& x_1, x_2, x_3, x_4, x_5 \ge 0
\end{align*}

\subsection*{Simplex Phase II}
\begin{align*}
\text{max. } & z = -20 + {\color{red} 2x_1} - x_4 \\
\text{s.t. } & x_2 = 10 + x_1 + \frac{1}{2}x_4 &&\big| x_1 \le 10 \\
& x_3 = 10 - 3x_1 - x_4 &&\big| x_1 \le \frac{10}{3} \\
& x_5 = 15 - {\color{red} 5x_1} - \frac{1}{2}x_4 &&\big| \color{red} x_1 \le 3 \to \text{minimale Beschränkung} \\
& x_1, x_2, x_3, x_4, x_5 \ge 0
\end{align*}

\subsection*{Nach Umformen und Einsetzen:}
\begin{align*}
\text{max. } & {\color{teal} z = -14} - \frac{2}{10}x_4 - \frac{2}{5}x_5 \\
\text{s.t. } & {\color{teal} x_1 = 3} - \frac{1}{10}x_4 - \frac{1}{5}x_5 \\
& {\color{teal} x_2 = 13} + \frac{2}{5}x_4 - \frac{1}{5}x_5 \\
& x_3 = 1 - \frac{7}{10}x_4 + \frac{6}{5}x_5 \\
& x_1, x_2, x_3, x_4, x_5 \ge 0
\end{align*}

\subsection*{Lösung:}
\begin{align*}
z &= -14 \\
x_1 &= 3 \\
x_2 &= 13
\end{align*}


\section*{Aufgabe 2}
\subsection*{Ausgangsmodell}
\begin{align*}
\text{max. } & z = 5x_1 + x_2 - 2x_3 \\
\text{s.t. } & x_1 + 2x_2 + 3x_3 \le 25 \\
& 5x_1 + 2x_3 = 25 \\
& 3x_1 - 8x_2 \ge 30 \\
& x_1, x_2, x_3 \ge 0
\end{align*}

$\to$ Keine freien Variablen. \\

\subsection*{Gleichungen zu Ungleichungen}
\begin{align*}
\text{max. } & z = 5x_1 + x_2 - 2x_3 \\
\text{s.t. } & x_1 + 2x_2 + 3x_3 \le 25 \\
& 5x_1 + 2x_3 \ge 25 \\
& 5x_1 + 2x_3 \le 25 \\
& 3x_1 - 8x_2 \ge 30 \\
& x_1, x_2, x_3 \ge 0
\end{align*}

$\to$ Bereits eine max-Zielfunktion. \\

\subsection*{Ungleichungen zu Gleichungen mit Schlupfvariablen}
\begin{align*}
\text{max. } & z = 5x_1 + x_2 - 2x_3 \\
\text{s.t. } & x_1 + 2x_2 + 3x_3 + x_4 = 25 \\
& 5x_1 + 2x_3 - x_5 = 25 \\
& 5x_1 + 2x_3 + x_6 = 25 \\
& 3x_1 - 8x_2 - x_7 = 30 \\
& x_1, x_2, x_3, x_4, x_5, x_6, x_7 \ge 0
\end{align*}

\subsection*{Nach BV umgestellt}
\begin{align*}
\text{max. } & z = 5x_1 + x_2 - 2x_3 \\
\text{s.t. } & x_4 = 25 - x_1 - 2x_2 - 3x_3 \\
& {\color{red} x_5 = -25} + 5x_1 + 2_x3 \color{red} \longrightarrow \text{Ungültig} \\
& x_6 = 25 - 5x_1 - 2x_3 \\
& {\color{red} x_7 = -30} + 3x_1 - 8x_2 \color{red} \longrightarrow \text{Ungültig} \\
& x_1, x_2, x_3, x_4, x_5, x_6, x_7 \ge 0
\end{align*}

$\to$ Simplex I notwendig.

\subsection*{Neue Zielfunktion \emph{s}}
\begin{align*}
\text{max. } s &= x_5 + x_7 \\
&= (-25 + 5x_1 + 2_x3) + (-30 + 3x_1 - 8x_2) \\
&= -55 + {\color{red} 8x_1} - 8x_2 + 2x_3
\end{align*}
\begin{align*}
\text{s.t. } & z = 0 + 5x_1 + x_2 - 2x_3 \\
& x_4 = 25 - x_1 - 2x_2 - 3x_3 &&\big| x_1 \uparrow 25 \\
& x_5 = -25 + {\color{red} 5x_1} + 2_x3 &&\big| \color{red} x_1 \uparrow 5 \\
& x_6 = 25 - 5x_1 - 2x_3 &&\big| x_1 \uparrow 5 \\
& x_7 = -30 + 3x_1 - 8x_2 &&\big| x_1 \uparrow 10 \\
& x_1, x_2, x_3, x_4, x_5, x_6, x_7 \ge 0
\end{align*}

\subsection*{Nach Umformen und Einsetzen - Iteration 1}
\begin{align*}
\text{max. } & s = -15 - 8x_2 - \frac{6}{5}x_3 + {\color{red} \frac{8}{5}x_5}\\
\text{s.t. } & z = 25 + x_2 - 4x_3 +x_5 \\
& x_1 = 5 - \frac{2}{5}x_3 + \frac{1}{5}x_5 &&\big| x_5 \uparrow \infty \\
& x_4 = 20 - 2x_2 - \frac{13}{5}x_3 - \frac{1}{5}x_5 &&\big| x_5 \uparrow 100 \\
& x_6 = 0 - {\color{red} x_5} &&\big| \color{red} x_5 \uparrow 0 \\
& x_7 = -15 - 8x_2 - \frac{6}{5}x_3 + \frac{3}{5}x_5 &&\big| x_5 \uparrow 25
\end{align*}

\subsection*{Nach Umformen und Einsetzen - Iteration 2}
\begin{align*}
\text{max. } & s = -15 - 8x_2 - \frac{6}{5}x_3 - \frac{8}{5}x_6 \color{red} \longrightarrow \text{Kein positiver Koeffizient mehr} \\
\text{s.t. } & z = 25 + x_2 - 4x_3 +x_5 \\
& x_1 = 5 - \frac{2}{5}x_3 - \frac{1}{5}x_6 \\
& x_4 = 20 - 2x_2 - \frac{13}{5}x_3 + \frac{1}{5}x_6 \\
& x_5 = 0 - x_6 \\
& {\color{red} x_7 = -15} - 8x_2 - \frac{6}{5}x_3 - \frac{3}{5}x_6 \color{red} \longrightarrow \text{Immer noch ungültig}
\end{align*}

$\to$ Abbruchbedingung des Simplex-Algorithmus ist erfüllt, es kann keine zulässige Lösung gefunden werden. \\

\vspace{4mm}
\includegraphics[width=.7\linewidth]{src/unzulaessig.png}

\section*{Aufgabe 3}
\subsection*{a)}
$x_{BA}, x_{HI}, x_{ER}, x_{Row3}, x_{Row6}$, da diese für die Lösung nicht auf 0 gesetzt wurden.

\subsection*{b)}
Der Umsatz pro Tag steigt durch den höheren Preis.
Die Produktionsmengen bleiben unverändert, da bereits die nach den Restriktionen maximal mögliche Menge Bananeneis produziert wird.

\subsection*{c)}
Ja, es bleiben Zutaten für 7,5 Kugeln / 750g Bananeneis (\emph{Row 3}) über.
Es bleiben Zutaten für 40 Kugeln / 4000g Himbeereis (\emph{Row 6}) über.

\subsection*{d)}
$x_{BA}$ wird bereits maximal produziert.\\
Zulässige Erhöhung: $+ \infty$. \\
Zulässige Reduktion: $-0,1$ € / Kugel. \\

\includegraphics[width=.8\linewidth]{src/zielfunktion.png}

\subsection*{e)}

Die Menge der Bananen darf unendlich erhöht werden. Sie darf um die für 7,5 Kugeln notwendige Menge reduziert werden (300g Bananen). \newline

Die Menge der Erdbeeren darf nicht erhöht werden. Sie darf um die Menge für 20 Kugeln Eis reduziert werden.

\includegraphics[width=.8\linewidth]{src/restriktionen.png}

\end{document}