\documentclass[a4paper,11pt]{article}

\usepackage[utf8]{inputenc} % Unicode support (Umlauts etc.)
\usepackage[ngerman]{babel} % Change hyphenation rules
\usepackage{ziffer} % , können in Zahlen verwendet werden ohne Formatierung kaputt zu machen
\usepackage[top=30mm,right=20mm,bottom=15mm,left=25mm,includefoot,headheight=32pt]{geometry} % Seitenränder

\usepackage{lmodern,textcomp} % The package supports the Text Companion fonts, which provide many text symbols (benötigt für €)
\usepackage[fleqn]{amsmath} % Formatierte Gleichungen
\usepackage{graphicx} % Grafiken
\usepackage{xcolor} % Farbe in Text
\usepackage{fancyhdr} % Seitenstil mit Kopfzeile etc.

\usepackage{cases} % Fallunterscheidungen mathematisch uebereinander

\usepackage{floatflt}

\pagestyle{fancy}
\fancyhf{}
\lhead{
    Lösung \\
    Übungsblatt 6
}
\rhead{Gruppe 3 \\Nils \textbf{Hodys}, Sascha \textbf{Majewsky}}
\rfoot{Seite \thepage}

\setlength{\parindent}{0cm} % Keine Einrückung der 1. Zeile eines Absatzes

\begin{document}

\raggedright % Alles Linksbündig

\section*{Aufgabe 1}
\subsection*{1.}
\begin{align*}
& A \land \neg (\neg B \to C) & \\
\Leftrightarrow & A \land \neg (B \lor C) && \big| \text{?} \\
\Leftrightarrow & A \land \neg B \land \neg C && \big| \text{De-Morgan} \\
\end{align*}

\subsection*{2.}
\begin{align*}
& A \to B \land \neg (C \lor D) & \\
\Leftrightarrow & \neg A \lor B \land \neg (C \lor D) && \big| \text{?} \\
\Leftrightarrow & \neg A \lor B \land \neg C \land \neg D && \big| \text{} \\
\end{align*}

\subsection*{3.}
\begin{align*}
& \neg(\neg A \land B) \lor (C \leftrightarrow D) & \\
\Leftrightarrow & \neg(\neg A \land B) \lor ((C \to D) \land (D \to C)) && \big| \text{?} \\
\Leftrightarrow & \neg(\neg A \land B) \lor ((\neg C \lor D) \land (\neg D \lor C)) && \big| \text{?} \\
\Leftrightarrow & A \lor \neg B \lor ((\neg C \lor D) \land (\neg D \lor C)) && \big| \text{De-Morgan} \\
\end{align*}


\section*{Aufgabe 2}
\begin{align*}
\textbf{1. } & P2 \to P3 \\
\Leftrightarrow & \neg P2 \lor P3 \\
\Rightarrow  & (1 - P2) + P3 \ge 1 \\ \\
%
\textbf{2. } & P4 \land P1 \\
\Rightarrow  & P4 + P1 = 2 \\ \\
%
\textbf{3. } & P7 \to P1 \\
\Leftrightarrow & \neg P7 \lor P1 \\
\Rightarrow  & (1 - P7) + P1 \ge 1 \\ \\
%
\textbf{4. } & \neg P6 \\
\Rightarrow & P6 = 0 \\ \\
%
\textbf{5. } & P5 \leftrightarrow (P4 \lor P3) \\
\Leftrightarrow & (P5 \to (P4 \lor P3)) \land ((P4 \lor P3) \to P5) \\
\Leftrightarrow & (\neg P5 \lor (P4 \lor P3)) \land (\neg (P4 \lor P3) \lor P5) \\
\Leftrightarrow & (\neg P5 \lor P4 \lor P3) \land (\neg P4 \land \neg P3 \lor P5) \\
\Rightarrow & \left\{\begin{array}{l}
                    -P5 + P4 + P3 \ge 0 \\
                    -P3 -P4 + P5 \ge 1 \\
                \end{array}\right.
\end{align*}


\section*{Aufgabe 3}

\subsection*{Binäre Entscheidungsvariablen}
\begin{align*}
    A &= \begin{cases}
        1, & \text{Gebiet A wird erschlossen} \\
        0, & \text{Gebiet A wird nicht erschlossen}
    \end{cases} \\
    B &= \begin{cases}
        1, & \text{Gebiet B wird erschlossen} \\
        0, & \text{Gebiet B wird nicht erschlossen}
    \end{cases} \\
    C &= \begin{cases}
        1, & \text{Gebiet C wird erschlossen} \\
        0, & \text{Gebiet C wird nicht erschlossen}
    \end{cases} \\
    D &= \begin{cases}
        1, & \text{Gebiet D wird erschlossen} \\
        0, & \text{Gebiet D wird nicht erschlossen}
    \end{cases} \\
    E &= \begin{cases}
        1, & \text{Gebiet E wird erschlossen} \\
        0, & \text{Gebiet E wird nicht erschlossen}
    \end{cases} \\
\end{align*}


\pagebreak

\subsection*{Boolsche Restriktionen}
$B \to A$ \\
$C \to (A \land B)$ \\
$E \to D$ \\

\subsection*{Algebraische Restriktionen}
\textit{In Tagen Bauzeit} \newline

$80A + 200B + 110C + 90D + 70E \le 500$

\subsubsection*{Algebraisch umgeformte boolsche Restriktionen}
$B - A \le 0$ \\
$A - C \ge 0$ \\
$B - C \ge 0$ \\
$E - D \le 0$ \\


\subsection*{Zielfunktion}
\textit{In TEuro} \newline

\begin{multline*}
\text{max. Gewinn } z = 1100A + 1200B + 2400C + 1200D + 1300E\\ - (1300A + 1600B + 1200C + 1000D + 900E)
\end{multline*}

\end{document}