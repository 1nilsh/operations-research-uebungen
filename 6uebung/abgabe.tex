\documentclass[a4paper,11pt]{article}

\usepackage[utf8]{inputenc} % Unicode support (Umlauts etc.)
\usepackage[ngerman]{babel} % Change hyphenation rules
\usepackage{ziffer} % , können in Zahlen verwendet werden ohne Formatierung kaputt zu machen
\usepackage[top=30mm,right=20mm,bottom=15mm,left=25mm,includefoot,headheight=28pt]{geometry} % Seitenränder

\usepackage{lmodern,textcomp} % The package supports the Text Companion fonts, which provide many text symbols (benötigt für €)
\usepackage[fleqn]{amsmath} % Formatierte Gleichungen
\usepackage{graphicx} % Grafiken
\usepackage{xcolor} % Farbe in Text
\usepackage{fancyhdr} % Seitenstil mit Kopfzeile etc.

\usepackage{cases} % Fallunterscheidungen mathematisch uebereinander

\pagestyle{fancy}
\fancyhf{}
\lhead{Lösung \\Übungsblatt 6}
\rhead{Gruppe 3 \\Nils \textbf{Hodys}, Sascha \textbf{Majewsky}}
\rfoot{Seite \thepage}

\setlength{\parindent}{0cm} % Keine Einrückung der 1. Zeile eines Absatzes

\begin{document}

\raggedright % Alles Linksbündig

\section*{Aufgabe 1}
\subsection*{1.}
\begin{align*}
& A \land \neg (\neg B \implies C) & \\
\text{wird zu } & A \land \neg (B \lor C) && \big| \text{?} \\
\text{wird zu } & A \land \neg B \land \neg C && \big| \text{De-Morgan} \\
\end{align*}

\subsection*{2.}
\begin{align*}
& A \implies B \land \neg (C \lor D) & \\
\text{wird zu } & \neg A \lor B \land \neg (C \lor D) && \big| \text{?} \\
\text{wird zu } & \neg A \lor B \land \neg C \land \neg D && \big| \text{} \\
\end{align*}

\subsection*{3.}
\begin{align*}
& \neg(\neg A \land B) \lor (C \iff D) & \\
\text{wird zu } & \neg(\neg A \land B) \lor ((C \implies D) \land (D \implies C)) && \big| \text{?} \\
\text{wird zu } & \neg(\neg A \land B) \lor ((\neg C \lor D) \land (\neg D \lor C)) && \big| \text{?} \\
\text{wird zu } & A \lor \neg B \lor ((\neg C \lor D) \land (\neg D \lor C)) && \big| \text{De-Morgan} \\
\end{align*}


\section*{Aufgabe 2}
\begin{align*}
\textbf{1. }& P2 \implies P3 \\
            & \neg P2 \lor P3 \\ \\
\textbf{2. }& P4 \land P1 \\ \\
\textbf{3. }& P7 \implies P1 \\
            & \neg P7 \lor P1 \\ \\
\textbf{4. }& \neg P6 \\ \\
\textbf{5. }& P5 \implies (P4 \lor P3) \\
            & \neg P5 \lor (P4 \lor P3) \\
            & P5 \land \neg (P4 \lor P3) \\
            & P5 \land (\neg P4 \land \neg P3)
\end{align*}


\section*{Aufgabe 3}

\end{document}