\documentclass[a4paper,11pt]{article}

\usepackage[utf8]{inputenc} % Unicode support (Umlauts etc.)
\usepackage[ngerman]{babel} % Change hyphenation rules
\usepackage{ziffer} % , können in Zahlen verwendet werden ohne Formatierung kaputt zu machen
\usepackage[top=30mm,right=20mm,bottom=15mm,left=25mm,includefoot,headheight=32pt]{geometry} % Seitenränder

\usepackage{lmodern,textcomp,amsfonts} % The package supports the Text Companion fonts, which provide many text symbols
\usepackage[fleqn]{amsmath} % Formatierte Gleichungen
\usepackage{graphicx} % Grafiken
\usepackage{xcolor} % Farbe in Text
\usepackage{fancyhdr} % Seitenstil mit Kopfzeile etc.
\usepackage{cases} % Fallunterscheidungen mathematisch übereinander

\usepackage{tikz} % Graphen
\usetikzlibrary{positioning}

\pagestyle{fancy}
\fancyhf{}
\lhead{
    Lösung \\
    Übungsblatt 10
}
\rhead{Gruppe 3 \\Nils \textbf{Hodys}, Sascha \textbf{Majewsky}}
\rfoot{Seite \thepage}

\setlength{\parindent}{0cm} % Keine Einrückung der 1. Zeile eines Absatzes

\begin{document}

\raggedright % Alles Linksbündig
\setlength{\mathindent}{0cm} % align* nicht eingerückt

\section*{Aufgabe 1}

\begin{tikzpicture}[inner sep=3mm, node distance=4cm and 1cm, thick, main/.style = {draw, circle}]
  \node[main] (1) []            { A };
  \node[main] (2) [below of=1]  { B };
  \node[main] (3) [right of=2]  { C }; 
  \node[main] (4) [right of=1]  { D };
  \node[main] (5) [right of=4]  { E }; 
  \node[main] (6) [right of=3]  { F };

  \draw[->] (1) -- node[midway] {7 (1, 5)} (2);
  \draw[->] (1) -- node[midway] {8 (4, 11)} (4);

  \draw[->] (2) -- node[midway] {3 (1, 6)} (3);

  \draw[->] (3) -- node[midway] {6 (1, 5)} (5);
  \draw[->] (3) -- node[midway] {11 (0, 7)} (6);

  \draw[->] (4) -- node[midway] {-5 (3, 8)} (3);
  \draw[  ] (4) -- node[midway] {2 (0, 4)} (5);

  \draw[->] (5) -- node[midway] {9 (0, 6)} (6);
\end{tikzpicture}


\section*{Aufgabe 2}

\begin{tikzpicture}[inner sep=3mm, node distance=4cm and 1cm, thick, main/.style = {draw, circle}]
  \node[main] (1) []            { S };
  \node[main] (2) [below of=1]  { 2 };
  \node[main] (3) [right of=2]  { 3 }; 
  \node[main] (4) [right of=1]  { 1 };
  \node[main] (5) [right of=4]  { 4 }; 
  \node[main] (6) [right of=3]  { 5 };
  \node[main] (7) [right of=5]  { T };

  \draw[->] (1) -- node[midway] {5} (2);
  \draw[->] (1) -- node[midway] {6} (4);

  \draw[->] (2) -- node[midway] {4} (3);

  \draw[->] (3) -- node[midway] {3} (5);
  \draw[->] (3) -- node[midway] {2} (6);

  \draw[->] (4) -- node[midway] {7} (3);
  \draw[->] (4) -- node[midway] {6} (5);
  \draw[->] (4) to [looseness=1] node[midway] {7} (7);

  \draw[->] (5) -- node[midway] {6} (6);
  \draw[->] (5) -- node[midway] {4} (7);

  \draw[->] (6) -- node[midway] {5} (7);
\end{tikzpicture}


\section*{Aufgabe 3}

\begin{tikzpicture}[inner sep=3mm, node distance=4cm and 1cm, thick, main/.style = {draw, circle}]
  \node[main] (1) []            { S };
  \node[main] (2) [below of=1]  { 2 };
  \node[main] (3) [right of=2]  { 3 }; 
  \node[main] (4) [right of=1]  { 1 };
  \node[main] (5) [right of=4]  { 4 }; 
  \node[main] (6) [right of=3]  { 5 };
  \node[main] (7) [right of=5]  { T };

  \draw[->] (1) -- node[midway] {1, 5} (2);
  \draw[->] (1) -- node[midway] {2, 6} (4);

  \draw[->] (2) -- node[midway] {1, 4} (3);

  \draw[->] (3) -- node[midway] {2, 3} (5);
  \draw[->] (3) -- node[midway] {4, 2} (6);

  \draw[->] (4) -- node[midway] {2, 7} (3);
  \draw[->] (4) -- node[midway] {1, 6} (5);
  \draw[->] (4) to [looseness=1] node[midway] {4, 7} (7);

  \draw[->] (5) -- node[midway] {1, 6} (6);
  \draw[->] (5) -- node[midway] {2, 4} (7);

  \draw[->] (6) -- node[midway] {3, 5} (7);
\end{tikzpicture}

\end{document}