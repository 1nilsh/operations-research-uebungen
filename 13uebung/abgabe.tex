\documentclass[a4paper,11pt]{article}

\usepackage[utf8]{inputenc} % Unicode support (Umlauts etc.)
\usepackage[ngerman]{babel} % Change hyphenation rules
\usepackage{ziffer} % , können in Zahlen verwendet werden ohne Formatierung kaputt zu machen
\usepackage[top=30mm,right=20mm,bottom=15mm,left=25mm,includefoot,headheight=32pt]{geometry} % Seitenränder

\usepackage{lmodern,textcomp,amsfonts} % The package supports the Text Companion fonts, which provide many text symbols
\usepackage[fleqn]{amsmath} % Formatierte Gleichungen
\usepackage{graphicx} % Grafiken
\usepackage{xcolor} % Farbe in Text
\usepackage{fancyhdr} % Seitenstil mit Kopfzeile etc.
\usepackage{cases} % Fallunterscheidungen mathematisch übereinander

\pagestyle{fancy}
\fancyhf{}
\lhead{
    Lösung \\
    Übungsblatt 13
}
\rhead{Gruppe 3 \\Nils \textbf{Hodys}, Sascha \textbf{Majewsky}}
\rfoot{Seite \thepage}

\setlength{\parindent}{0cm} % Keine Einrückung der 1. Zeile eines Absatzes

\begin{document}

\raggedright % Alles Linksbündig
\setlength{\mathindent}{0cm} % align* nicht eingerückt

\section*{Aufgabe 1}
\subsection*{a) ADD-Algorithmus}
\subsubsection*{Schritt 0}
\begin{tabular}{l|l|l|l|l|l||l}
      & PA & NA & MA & TO & $f_i$ & $\sum$ \\ \hline
  LY  & 110 & 130 & 110 & 80 & 310 & 740 \\ \hline
  NI  & 190 & 230 & 60 & 70 & 260 & 810 \\ \hline
  MO  & 160 & 130 & 40 & 30 & 280 & \textbf{640} \\ \hline
  PA  & 5 & 120 & 150 & 200 & 290 & 765 \\ 
\end{tabular} \newline

Ergebnis: Eröffne Standort MO endgültig. \\
$I_0^{VL} = \{LY, NI, PA\}; I_0 = \{ \}; I_1 = \{MO\}; z = 640$

\subsubsection*{Schritt 1}
Berechne Matrix $W$ mit: $w_{ij} = \text{max} \{c_{MO \to j} - c_{i \to j}, 0\}$ für alle $i$ aus $I_0^{VL}$ \\

\begin{tabular}{l|l|l|l|l||l|l|l}
    & PA & NA & MA & TO & $\sum w_i$ & $f_i$ & $w_i - f_i$ \\ \hline
LY  & 50 & 0 & 0 & 0 & 50 & 310 & 0 \\ \hline
NI  & 0 & 0 & 0 & 0 & 0 & 260 & 0 \\ \hline
PA  & 155 & 10 & 0 & 0 & 165 & 290 & 0 \\ 
\end{tabular} \newline

Ergebnis: Verbiete Standorte LY, NI, PA endgültig. \\
$I_0 = \{ LY, NI, PA \}; I_1 = \{ MO \}; z = 640$ \\

~\newline

\textit{Alle Kunden werden durch das Lager Montpellier beliefert. Es entstehen Kosten von 640 TEuro (Zielfunktionswert).} \\

\subsection*{b) DROP-Algorithmus}
\begin{tabular}{l|l|l|l|l|l}
      & PA & NA & MA & TO & $f_i$ \\ \hline
  LY  & 110 & 130 & 110 & 80 & 310 \\ \hline
  NI  & 190 & 230 & 60 & 70 & 260 \\ \hline
  MO  & 160 & 130 & 40 & 30 & 280 \\ \hline
  PA  & 5 & 120 & 150 & 200 & 290 \\ 
\end{tabular} \newline

$z = (310 + 260 + 280 + 290) + (5 + 120 + 40 + 30) = 1335$ \\

\subsubsection*{Iteration 1}
Wenn Standort ... verboten wird: \newline

LY: $z = (260 + 280 + 290) + (5 + 120 + 40 + 30) = 1025$ $\to$\textbf{Verbesserung: 310} \\
NI: $z = (310 + 280 + 290) + (5 + 120 + 40 + 30) = 1075$ $\to$Verbesserung: 260 \\
MO: $z = (310 + 260 + 290) + (5 + 120 + 60 + 70) = 1115$ $\to$Verbesserung: 220 \\
PA: $z = (310 + 260 + 280) + (110 + 130 + 40 + 30) = 1160$ $\to$Verbesserung: 175 \\
~\newline
$\longrightarrow$ Standort LY endgültig verbieten. \\
$\longrightarrow$ $I_0 = \{LY\}; I_1 = \{ \}; I_1^{VL} = \{NI, MO, PA\}$ \\
~\newline

\begin{tabular}{l|l|l|l|l|l}
    & PA & NA & MA & TO & $f_i$ \\ \hline
NI  & 190 & 230 & 60 & 70 & 260 \\ \hline
MO  & 160 & 130 & 40 & 30 & 280 \\ \hline
PA  & 5 & 120 & 150 & 200 & 290 \\ 
\end{tabular} \newline

$z = (260 + 280 + 290) + (5 + 120 + 40 + 30) = 1025$ \\

\subsubsection*{Iteration 2}
Wenn Standort ... verboten wird: \newline

NI: $z = (280 + 290) + (5 + 120 + 40 + 30) = 765$ $\to$\textbf{Verbesserung: 260} \\
MO: $z = (260 + 290) + (5 + 120 + 60 + 70) = 805$ $\to$Verbesserung: 220 \\
PA: $z = (260 + 280) + (160 + 130 + 40 + 30) = 900$ $\to$Verbesserung: 125 \\
~\newline
$\longrightarrow$ Standort NI endgültig verbieten. \\
$\longrightarrow$ $I_0 = \{LY, NI\}; I_1 = \{ \}; I_1^{VL} = \{MO, PA\}$
~\newline

\begin{tabular}{l|l|l|l|l|l}
    & PA & NA & MA & TO & $f_i$ \\ \hline
MO  & 160 & 130 & 40 & 30 & 280 \\ \hline
PA  & 5 & 120 & 150 & 200 & 290 \\ 
\end{tabular} \newline

$z = (280 + 290) + (5 + 120 + 40 + 30) = 765$ \\

\subsubsection*{Iteration 3}
Wenn Standort ... verboten wird: \newline

MO: $z = (290) + (5 + 120 + 150 + 200) = 765$ $\to$Verbesserung: 0 \\
PA: $z = (280) + (160 + 130 + 40 + 30) = 640$ $\to$\textbf{Verbesserung: 125} \\
~\newline
$\longrightarrow$ Standort PA endgültig verbieten. \\
$\longrightarrow$ $I_0 = \{LY, NI, PA\}; I_1 = \{ \}; I_1^{VL} = \{MO\}$
~\newline

\begin{tabular}{l|l|l|l|l|l}
    & PA & NA & MA & TO & $f_i$ \\ \hline
MO  & 160 & 130 & 40 & 30 & 280 \\
\end{tabular} \newline

$z = (280) + (160 + 130 + 40 + 30) = 640$ \\

\subsubsection*{Iteration 4}
Wenn Standort ... verboten wird: \newline

MO: $z = (290) + (5 + 120 + 150 + 200) = 765$ $\to$Verbesserung: 0 \\
~\newline
$\longrightarrow$ Fertig, da keine Verbesserung durch weiteres Verbieten mehr möglich. Die vorläufig eröffneten Standorte werden endgültig eröffnet. \\
$\longrightarrow$ $I_0 = \{LY, NI, PA\}; I_1 = \{MO\}$

~\newline

\textit{Alle Kunden werden durch das Lager Montpellier beliefert. Es entstehen Kosten von 640 TEuro (Zielfunktionswert).} \\

\section*{Aufgabe 2}
\subsection*{Mengen}
Menge der potenziellen Standorte $J=\{1,...,n\}$ \\
Menge der Kunden $I=\{1,...,m\}$ \\

\subsection*{Parameter}
\begin{align*}
    j &: ~\text{Einzelner Standort} ~ j \in J \\
    i &: ~\text{Kundengruppe} ~ i \in I \\
    c_{ij} &: ~\text{Kosten für Transport} \\
    k &: ~\text{Höchstzahl Standorte} \\
    f_j &: ~\text{Errichtungskosten für Standort j} \\
    l_j &: ~\text{Untere Schranke Anzahl belieferter Kunden aus Standort j} \\
    u_j &: ~\text{Obere Schranke Anzahl belieferter Kunden aus Standort j} \\
    a_i &: ~\text{Anzahl Kunden in Gruppe i} \\
    B &: ~\text{Obere Schranke Errichtungskosten für alle Standorte} \\
\end{align*}

\subsection*{Variablen}
\begin{align*}
    y_j &: \begin{cases}
        1, & \text{Standort j wird Errichtet} \\
        0, & \text{Standort j wird nicht Errichtet}
    \end{cases} \\
    x_{ji} &: \{ x ~|~ 0 \le x \le 1 \} ~\text{wobei $x_{ij}$ = 1, wenn Kunde i von dem pot. Standort j vollständig beliefert wird} \\
\end{align*}

\subsection*{Zielfunktion}
\begin{align*}
    \text{min.} ~& \sum_{j=1}^{n}\sum_{i=1}^{m}c_{ij} x_{ij} \\
    ~\\
    \text{s.t.} ~& x_{ji} \le y_j ~\text{für alle i und j} &&|~ \text{Es wird nur aus errichteten Standorten geliefert}  \\
    & \sum_{j=1}^{n} f_j y_j \le B &&|~ \text{Kosten für errichtete Standorte kleiner als obere Schranke B} \\
    & \sum_{j=1}^{n} y_j \le k &&|~ \text{Anzahl errichteter Standorte kleiner als Höchstzahl k} \\
    & \sum_{i=1}^{m} y_j x_{ji} a_i \ge l_j ~\text{für alle j} &&|~ \text{Aus errichteten Standorten werden mindestens l Kunden beliefert} \\
    & \sum_{i=1}^{m} y_j x_{ji} a_i \le u_j ~\text{für alle j} &&|~ \text{Aus errichteten Standorten werden höchstens u Kunden beliefert} \\
    & y_j \in \{0, 1\} &&|~ \text{Binärvariable} \\
    & x_ji \ge 0 &&|~ \text{NNB} \\
\end{align*}

\end{document}